\subsection{Prerequisites}
This part will be implemented in the future and are for now only some good
ideas we have for this project. We are first focused on the achievement of the simple
control of this car.

\subsection{Mission}
We have establish a simple command for our kart based on image processing. This
command law is quite easy to setup and fully functionnal as we could see. However
this command is efficient only if the camera is able to see a line. But actually
the car could be unable to see a line, particularly when the car will accelerate
in order to run a lap as quick as possible. Our strategy on this topic is to run
a first lap very slow and to drop virtual gps tags on a map while the car is following
the line. So after a lap we will get a whole map of the circuit and at any given times
the car will have to determine if a line is available on the camera, and if the car
could follow this line, else the car will have to reach this line again with the help
of the gps tags. Then after the first lap, the car will be able to accelerate and for instance some times
lost the line because it will be able to reach again the circuit with the gps tags.

\subsection{GUI}
We found quit interresting to have a beautifull Graphical User Interface for our
car. We are thinking about a frontend solution in order to show dynamical parameters
such as acceleration and seep of the car, and a map to show the position with the 
classical circle to represent the incertainty of this position. Moreover it could
be interresting to have the camera output available on this GUI. The easiest way to
display a map and some gnss coordinates on python is to use the \textit{folium} package
based on the \textit{OpenStreetMap} map. We are now working on a bind between \textit{ROS},
\textit{folium} and the web framework \textit{Django} to sum up these informations on
a state webpage. However, we don't own any GNSS or Inertial Units to prepare hardware
nodes. That is also not our prioriy, but it can be a very usefull tool.

