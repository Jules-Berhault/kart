\section{Hardware}

\subsection{Introduction}

This section will present the hardware architecture of this project. The assembly must meet the constraints of the embedded
systems, i.e. constraints of size, weight and power consumption. As shown in
figure 1, the car is based on the Raspberry Pi 3B+ board and has a rear engine
managed by an ESC and a servo motor to control the steering at the front. Finally,
as far as sensors are concerned, we have at our disposition an \textit{Inertial Unit}, a 
\textit{Global Navigation Satellite System} (GNSS) and a \textit{Camera}
interfaced to the dedicated CSI port.


\subsection{Main Board}

On this \textit{Raspberry Pi} we will need to install an operating system in order
to run our programs. Furthermore, the connectivity of this board is suited to our
problem since we have USB ports as well as access to the GPIO pins that allow us
to control several devices.

\subsection{Sensors}
We have at our service for this project 3 different sensors that will allow the robot
to acquire information about its state: 

\paragraph{GNSS}
Provides access to the latitude and longitude of the car. In combination with a Kalman filter, it satisfies the need to know the position of the car. It could also be useful in cases where the car loses track of the line, in order to come back to it.

\paragraph{Inertial Unit}
Allows access to the car's acceleration. This is useful to limit the car's acceleration and thus prevent the car from skidding in its path. However, we do not have it available at the moment, so we have not been able to implement it within this system.

\paragraph{Camera}
Because we are using a \textit{Raspberry Pi 3B+}, the camera we have choosen is the official 
camera which can be pluged on the dedicated port on the board. This sensor is
perfectly suited to our problem, because with an appropriated image processing
we will be able to detect the line and to correct the car trajectory.The setting is quite simple and is done via the \textit{raspi-config} menu. Moreover the ROS hardware node \textit{cv\_camera} that we found interfaces automatically with the camera connected to the dedicated CSI port. So we just have to get the image on a topic to do the image processing.

\subsection{Actuators}
In order to manage the direction and speed of the car, we need to be able to generate PWM signals. The first signal will control the ESC that manages the rear engine. The second signal to be generated will allow to manage the steering of the car via the control of the front servomotor.

On a \textit{Raspberry Pi} board there is several way to generate pwm signals.
The most of the time, these methods are software-based. It's better to use hardware 
generated pwm, because if there is a high CPU usage, interrupts
will not be correctly handled, the pwm duty cycle will not be very accurate and
the car will not be able for instance to follow a straight line, because the
bearing of the car is controled by a servomotor with a pwm signal.

According to this article \cite{pi_pwm}, there are two hardware PWM pins on a \textit{Raspberry Pi 3B+}. Some automation was required to set up the file system to allow the use of hardware PWM. So we had to use crontabs to keep the installation up and running.
After this installation the pwm is operational and allows us to fully control the rear engine and servo motor of the car.