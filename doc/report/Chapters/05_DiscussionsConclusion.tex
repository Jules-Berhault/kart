In conclusion, we have first established the requirements of this
system. The main objectives are to ensure that the car performs
laps of the track, regardless of its geometry, in autonomy and as
quickly as possible. Then the use of ROS allowed us to separate the 
functionalities in different nodes in order to facilitate the
development of the system. This allowed us to divide the work
but also to be more efficient in the development of the software
architecture of the car. Then we made a simulator that allowed
us to validate the different requirements of this system. It was
very useful in the current context which did not allow us to test
the car in a real environment. We were able to validate a minimum
speed criterion of $3 m.s^{-1}$, because we were able to push the speed
on the simulation up to $6 m.s^{-1}$. Then the range was checked as the 
car was able to finish the lap in simulation. However at the moment
the car must not lose sight of the line of sight, otherwise it
will not be able to steer anymore. That's why we would have liked
to use the GPS implementation we could have done. Indeed our idea
would have been to use the previous GPS positions that would have
been made in previous laps in order to be able to come back on the
line in case we lost it. This would have meant that at every moment
the car would have had a command to give to the actuators.

