\subsection{Explaination}

Based on the practical work made with Benoit Zerr, we got a realisitic and a track with a line in the middle of the track.
The simulated in VREP car is equiped with a camera. Our VREP simulator is interfaced with ROS through the piece of LUA code :
\texttt{rc\_car\_control\_ros.lua} et \texttt{rc\_car\_onboard\_cam.lua}.

The VREP camera send a through a topic \texttt{image}, a picture of the front of the car. 
Then the \texttt{camera\_node.py} subscriber process the image with OpenCV and return the position
of the center line in the windows, then it compute an error, we want the line in the middle of the windows. 
This error is sent through a topic \texttt{error}, the controller is suscribed to this topic,
it generate a command thanks to a PD command law which is sent to VREP. And now the car can loop forever, and achieve 
the perfect run in 1'15'' ! 🚙 🏁 🕙


Here is the video link : 


\url{https://www.youtube.com/watch?time_continue=10&v=_vIXo1TvG0w&feature=emb_logo} ⏯

Inspired from Benoit Zerr work : \url{https://www.ensta-bretagne.fr/zerr/dokuwiki/doku.php?id=vrep:create-rc-car-robot}
